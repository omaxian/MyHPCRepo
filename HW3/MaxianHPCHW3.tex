\documentclass[a4paper]{article}

%% Language and font encodings
\usepackage[english]{babel}
\usepackage[utf8x]{inputenc}
\usepackage[T1]{fontenc}
\usepackage{subfigure}

%% Sets page size and margins
\usepackage[a4paper,top=3cm,bottom=3.5cm,left=3cm,right=3cm,marginparwidth=1.75cm]{geometry}

%% Useful packages
\usepackage{amsmath}
\usepackage{graphicx}
\usepackage[colorlinks=true, allcolors=blue]{hyperref}
\newcommand{\norm}[1]{\left\lVert#1\right\rVert}
\newcommand{\bm}[1]{\mbox{\boldmath $ #1 $}}    % bold math mode


\title{HPC Homework 3}
\author{Ondrej Maxian}

\begin{document}
\maketitle

\section{Approximating special functions}
\textbf{Extra credit.} Consider an angle $\theta$ in radians. We write $\displaystyle \theta = x_1 - n\frac{\pi}{2}$, where $\displaystyle x_1 \in \left[-\frac{\pi}{4},\frac{\pi}{4}\right]$. Table \ref{tab:sintrick} shows how we can make use of symmetries to determine $\sin{\theta}$ using the rapidly converging Taylor series representations of $\sin{x_1}$ and $\cos{x_1}$. 

\begin{table}[ht]
\centering
\begin{tabular}{c|c}
$n$ (mod 4) & $\sin{\theta}$\\[2 pt] \hline
0 & $\sin{x_1}$ \\[2 pt]
1 & $\cos{x_1}$\\[2 pt]
2 & $-\sin{x_1}$\\[2 pt]
3 & $-\cos{x_1}$ 
\end{tabular}
\caption{Values of $\sin{\theta}$ in terms of $x_1$}
\label{tab:sintrick}
\end{table}

Let $z = \lfloor{\frac{n}{2}}\rfloor$. Then Table \ref{tab:sintrick} can be written concisely as
\begin{equation}
\sin{\theta} = (-1)^z (n+1 \textrm{ mod } 2)\sin{x_1} + (-1)^z  (n \textrm{ mod } 2)\cos{x_1}. 
\end{equation}
This function has been implemented in the methods \texttt{sin4\_taylor} and \texttt{sin4\_vec}. For $10^6$ values of $\theta \in [-\pi,\pi]$, the function \texttt{sin4\_taylor} takes about 80\% of the time of the built in function, and \texttt{sin4\_vec} takes 20\% of the time, an incredible speed-up. 

\section{Parallel scan in OpenMP}
I have parallelized the serial code and am running it on the CIMS computer, which has an Intel Core i7-6700 processor with 4 cores. The array has size $10^8$. Timings are reported in Table \ref{tab:omptime}. We see that the optimal number of threads is the same as the number of cores. 

\begin{table}[ht]
\centering
\begin{tabular}{c|c}
\# Threads & Time\\[2 pt] \hline
Serial & 0.371 \\[2 pt]
2 & 0.279 \\[2 pt]
4 & 0.181\\[2 pt]
6 & 0.189\\[2 pt]
8 & 0.192 
\end{tabular}
\caption{Timings for parallelized scan.}
\label{tab:omptime}
\end{table}


\end{document}